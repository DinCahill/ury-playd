U\+R\+Y playd ({\ttfamily playd} for short) is a minimal C++ audio player using \href{http://sox.sourceforge.net/}{\tt libsox} and \href{http://www.portaudio.com/}{\tt Port\+Audio}, developed by \href{http://ury.org.uk}{\tt University Radio York} (U\+R\+Y) and designed to be composable into bigger systems.

All code developed for {\ttfamily playd} is licenced under the \href{http://opensource.org/licenses/MIT}{\tt M\+I\+T licence} (see L\+I\+C\+E\+N\+C\+E.\+txt). Some code is taken from the \href{http://www.portaudio.com/}{\tt Port\+Audio} project (see L\+I\+C\+E\+N\+C\+E.\+portaudio).

\subsection*{Usage}

{\ttfamily playd D\+E\+V\+I\+C\+E-\/\+I\+D \mbox{[}A\+D\+D\+R\+E\+S\+S\mbox{]} \mbox{[}P\+O\+R\+T\mbox{]}}


\begin{DoxyItemize}
\item Invoking {\ttfamily playd} with no arguments lists the various device I\+Ds available to it.
\item Full protocol information is available on the Git\+Hub wiki.
\item On P\+O\+S\+I\+X systems, see the enclosed man page.
\end{DoxyItemize}

{\ttfamily playd} understands the following commands via its T\+C\+P/\+I\+P interface\+:


\begin{DoxyItemize}
\item {\ttfamily load \char`\"{}/full/path/to/file\char`\"{}} — Loads /full/path/to/file for playback;
\item {\ttfamily eject} — Unloads the current file;
\item {\ttfamily play} — Starts playback;
\item {\ttfamily stop} — Stops (pauses) playback;
\item {\ttfamily seek 1m} — Seeks one minute into the current file. Units supported include {\ttfamily h}, {\ttfamily m}, {\ttfamily s}, {\ttfamily ms}, {\ttfamily us} (micros), with {\ttfamily us} assumed if no unit is given.
\item {\ttfamily quit} — Closes {\ttfamily playd}.
\end{DoxyItemize}

\subsubsection*{Sending commands manually}

To connect directly to {\ttfamily playd} and issue commands to it, you can use \href{http://nc110.sourceforge.net/}{\tt netcat}\+:

```sh \section*{If you specified \mbox{[}A\+D\+D\+R\+E\+S\+S\mbox{]} or \mbox{[}P\+O\+R\+T\mbox{]}, replace localhost and 1350 respectively.}

\$ nc localhost 1350 ```

On Windows, using \href{http://www.chiark.greenend.org.uk/~sgtatham/putty/}{\tt Pu\+T\+T\+Y} in {\itshape raw mode} ({\bfseries not} Telnet mode) with {\itshape Implicit C\+R in every L\+F} switched on in the {\itshape Terminal} options should work.

{\bfseries Do {\itshape not} use a Telnet client (or Pu\+T\+T\+Y in telnet mode)!} {\ttfamily playd} will do weird things in the presence of Telnet-\/isms.

\subsection*{Features}


\begin{DoxyItemize}
\item Plays anything \href{http://sox.sourceforge.net/}{\tt libsox} can play (in practice, more esoteric formats might not work)
\item Seek (microseconds, seconds, minutes etc)
\item Frequently announces the current position
\item T\+C\+P/\+I\+P interface with text protocol
\item Deliberately not much else
\end{DoxyItemize}

\subsection*{Philosophy}


\begin{DoxyItemize}
\item Do one thing and do it well
\item Be hackable
\item Favour simplicity over performance
\item Favour simplicity over features
\item Let other programs handle the shinies
\end{DoxyItemize}

\subsection*{Compilation}

\subsubsection*{Requirements}


\begin{DoxyItemize}
\item \href{http://sox.sourceforge.net/}{\tt libsox} (1.\+14.\+1)
\item \href{https://github.com/joyent/libuv}{\tt libuv} (0.\+11.\+29)
\item \href{http://www.portaudio.com/}{\tt Port\+Audio} (19\+\_\+20140130)
\item A C++11 compiler (recent versions of \href{http://clang.llvm.org/}{\tt clang}, \href{https://gcc.gnu.org/}{\tt gcc}, and Visual Studio work)
\end{DoxyItemize}

Certain operating systems may need additional dependencies; see the O\+S-\/specific build instructions below.

\subsubsection*{P\+O\+S\+I\+X (G\+N\+U/\+Linux, B\+S\+D, O\+S X)}

{\ttfamily playd} comes with a G\+N\+U-\/compatible Makefile that can be used both to make and install.

To use the Makefile, you'll need \href{https://www.gnu.org/software/make/}{\tt G\+N\+U Make} and {\ttfamily pkg-\/config} (or equivalent), and pkg-\/config packages for Port\+Audio, libsox and libuv. We've tested building playd on Gentoo, Free\+B\+S\+D 10, and O\+S X, but other P\+O\+S\+I\+X-\/style operating systems should work.

Using the Makefile is straightforward\+:


\begin{DoxyItemize}
\item Ensure you have the dependencies above;
\item Read the {\ttfamily Makefile}, to see if any variables need to be overridden for your environment;
\item Run {\ttfamily make} (or whatever G\+N\+U Make is called on your O\+S; in Free\+B\+S\+D, for example, it'd be {\ttfamily gmake}), and, optionally, {\ttfamily sudo make install}. The latter will globally install playd and its man page.
\end{DoxyItemize}

\paragraph*{O\+S X}

All dependencies are available in \href{http://brew.sh}{\tt homebrew} -\/ it is highly recommended that you use it!

\subsubsection*{Free\+B\+S\+D (10+)}

Free\+B\+S\+D 10 and above come with {\ttfamily clang} 3.\+3 as standard, which should be able to compile {\ttfamily playd}. {\ttfamily gcc} is available through the Free\+B\+S\+D Ports Collection and package repositories.

You will need {\ttfamily gmake}, as {\ttfamily Makefile} is incompatible with B\+S\+D make. Sorry!

All of {\ttfamily playd}'s dependencies are available through both the Free\+B\+S\+D Ports Collection and standard package repository. (The Free\+B\+S\+D port for Port\+Audio doesn't build C++ bindings, but we bundle them anyway.) To install them as packages\+:

``` root\+:/ \# pkg install gmake sox libuv portaudio2 pkgconf ```

Then, run {\ttfamily gmake} ({\bfseries not} {\ttfamily make}), and, optionally, {\ttfamily gmake install} to install {\ttfamily playd} (as root)\+:

``` user\+:$\sim$/ \% gmake root\+:$\sim$/ \# gmake install ```

\subsubsection*{Windows}

\paragraph*{Visual Studio}

{\itshape For more information, see {\ttfamily \hyperlink{README_8VisualStudio_8md_source}{R\+E\+A\+D\+M\+E.\+Visual\+Studio.\+md}}.}

playd {\bfseries can} be built with Visual Studio (tested with 2013 Premium), but you will need to source and configure the dependencies manually. A Visual Studio project is provided, but will need tweaking for your environment.

\paragraph*{Min\+G\+W}

We haven't managed ourselves, but assuming you can build all the dependencies, (libsox is the difficult one), it should work fine.

\subsubsection*{Port\+Audio C++ Bindings}

If you have the Port\+Audio C++ bindings available, those may be used in place of the bundled bindings. This will happen automatically when using the Makefile, if the C++ bindings are installed as a pkg-\/config package.

{\bfseries Visual Studio users\+:} The Visual Studio 7.\+1 project supplied in the Port\+Audio source distribution for building the C++ bindings ({\ttfamily \textbackslash{}bindings\textbackslash{}cpp\textbackslash{}build\textbackslash{}vc7\+\_\+1\textbackslash{}static\+\_\+library.\+vcproj}) should work. If not, then use the bundled bindings.

\subsection*{Q\&A}

\subsubsection*{Why does this exist?}

It was originally written as an experiment when coming up with a new playout system for \href{http://ury.org.uk}{\tt University Radio York}.

\subsubsection*{Why is it named {\ttfamily playd}?}

It's short for \+\_\+\+\_\+play\+\_\+\+\_\+er \+\_\+\+\_\+d\+\_\+\+\_\+aemon.

\subsubsection*{Can I contribute?}

Certainly! We appreciate any and all pull requests in accordance with our philosophy. 